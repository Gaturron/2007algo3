%\section{Ejercicio 2}

%\subsection{insertar en \rbt}

%\subsection{insertar en \adi}


\begin{itemize}
\item \textit{prepararArregloEstanImagenes(int im)}

inicializa un arreglo en false\\
Costo $O(im)$

\item \textit{intervalosElementales(Lista(imagenes) img) $->$ Conj(puntos):res}

\begin{framed}
\begin{verbatimtab}[4]
intervalosElementales(Lista(imagenes) img) -> Conj(puntos):res
{
	desde 0 a Tam(img) hacer	// O(1) / itera im veces
	{
		res <- agregar(x_0)		// O(log(2im))
		res <- agregar(x_1)		// O(log(2im))
	}
	return res					// O(1)
}
\end{verbatimtab}
\end{framed}

Como ingresamos dos puntos por imagen a lo sumo el conjunto tendra $2im$, por lo que pudimos acotar la insercion $log(2im)$.\\
El costo total de este algoritmo es pues $O(im*log(im))$
            
\item \textit{insertar(IntervaloElemental ie)}

\item \textit{llenarIntervalosElementales(Lista(Imagen):imagenes, bool:X)}
\begin{framed}
\begin{verbatimtab}[4]          
llenarIntervalosElementales(Lista(Imagen):imagenes, bool:X)
{
	Conj(IntervaloElemental):puntos = intervalosElementales(imagenes, X);	
										// O(im*log(im))

	por cada imagen de "imagenes"		// itera im veces
		insertar(elemento)				// O(log(n))
}
\end{verbatimtab}
\end{framed}

Entonces la complejidad de insertar todos los intervalos elementales es\\
$O(im*log(im)) + O(im*log(n))$
            
\item \textit{insertarImagen(int indiceImagen, inicio, final, min, max, Nodo actual)}
      
\item \textit{insertarImagenes(Lista(Imagen):imagenes, bool:X)}
\begin{framed}
\begin{verbatimtab}[4]
insertarImagenes(Lista(Imagen):imagenes, bool:X)
{
	depeniendo de si es por X o por Y
		por cada imagnen de "imagenes"
			insertarImagen(i, inicio, fin, 0, ANCHO_ARBOL, raiz);
}
\end{verbatimtab}
\end{framed}

\item \textit{llenarArbol( Lista(Imagen):imagenes, bool:X) $->$ ArbolDeIntervalos}
\begin{framed}
\begin{verbatimtab}[4]          
llenarArbol( Lista(Imagen):imagenes, bool:X) -> ArbolDeIntervalos
{
	prepararArregloEstanImagenes(im);
	llenarIntervalosElementales(imagenes, X);
	insertarImagenes(imagenes, X);
}
\end{verbatimtab}
\end{framed}




\end{itemize}	

 
            
\begin{itemize}

\item \textit{buscarIndices(int punto, Nodo actual, conj(int))}
\begin{framed}
\begin{verbatimtab}[4]
buscarIndices(int punto, Nodo actual, conj(int))
{
	si no llegue al final del arbol
		res <- agregar(los indices que aparecen en el nodo)
		{
		si  punto es <= actual_numero
			buscarIndices(punto, Izq(actual), conj(int))
		si  punto es >= actual_numero
			buscarIndices(punto, Der(actual), conj(int))		
		}
}
\end{verbatimtab}
\end{framed}

Este es un algoritmo recursivo, que separa siempre el problema es dos partes iguales.\\
�Entra siempre por las dos ramas?... No\\
Existe a lo sumo solo un nodo en el arbol que tiene el mismo valor que el punto buscado, por lo tanto el algoritmo, hace a lo sumo dos caminos.\\
Que recorra uno o dos caminos es despreciable, por lo que su complejidad se puede expresar de la siguinte manera.
\[
\left\{
\displaystyle\matrix{
T(0) = d\hfill\cr
T(n) = T(n/2) + c + k_i
} 
\right.
\]
desarrollando una vez 
\[T(n) = T(n/4) + c + k_i + c + k_j\]
la formula general es
\[T(n) = T(n/2^h) + h*c + \displaystyle\sum_{i=1}^h k_i \]
con $h$ es la altura del arbol\\
como es un \rbt la altura es cercana a $log(n)$
\[T(n) = T(n/2^{log(n)}) + log(n)*c + \displaystyle\sum_{i=1}^h k_i\]
como a cada imagen la incuentro una sola vez por camino (pues si la encontre en un nodo no puedo escontrarla en sus hijos), habr� encontrado al final mis $k$ imagenes buscadas.
\[T(n) = T(n/n) + log(n)*c + k\]
\[T(n) = T(1) + log(n)*c + k\]
Esto da un complejidad total de $O(log(n) + k)$.

\item \textit{busqueda(x, y, Lista(Imagen):Imagenes\_levantadas, bool:X) $->$ Lista(Imagen)}
\begin{framed}
\begin{verbatimtab}[4]
busqueda(x, y, Lista(Imagen):Imagenes_levantadas, bool:X) -> Lista(Imagen):res
{
	dependiendo si es por X o por Y					// O(1)
		Lista(int):resIndices;						// O(1)
		buscarIndices(x, raiz, resIndices);			// O(log(n) + k)

		desde i=0 a Tam(resIndices) hacer			// itera k veces
		{			
			si esta entre los valores de y o x		// O(1)
				agregar imagen al resutado			// O(1)
			borrar del vector EstanImagenes			// O(1)
		}		
}
\end{verbatimtab}
\end{framed}

Este algoritmo es simple, consigue los indices de las imagenes que debe agregar segun una de las coordendas ($x$ o $y$) y vesifica luego que cumpla la otra coordenada tambien

Complejidad: $O(2 + log(n) + k + 3k) = O(log(n) + k)$

\end{itemize}



%\[
%T(n) = 
%\left\{
%\displaystyle\matrix{
%d & n=0\cr
%T(n/2) + c + k_i & n>0
%} 
%\right.
%\]
%\displaystyle\sum_{k=1}^N k^2 
% ${x+1 \brace x-1}$
%$\displaystyle\matrix{\strut\hbox{1}\hfill\cr\strut\hbox{2}\hfill\cr}$